\documentclass[11pt]{article}

\usepackage{geometry}                % See geometry.pdf to learn the layout options. There are lots.
\geometry{letterpaper}                   % ... or a4paper or a5paper or ... 

\usepackage[parfill]{parskip}    % Activate to begin paragraphs with an empty line rather than an indent
\usepackage{graphicx}
\usepackage{amssymb}
\usepackage{epstopdf}

\DeclareGraphicsRule{.tif}{png}{.png}{`convert #1 `dirname #1`/`basename #1 .tif`.png}

\begin{document}

\title{Diffusive Programming}
\author{Robert Philipp}
\date{\today}
\maketitle

%
% abstract
%
\begin{abstract}
This article describes an approach to task-oriented distributed programming that allows a developer to simply \emph{mark} a method to inform the distribution framework to distribute its execution. \emph{Diffusive programming} is defined through a set of principles that govern the action of distributing method-level execution and the topology of the distribution network. Through these principles, \emph{diffusive programming} allows the construction of distribution networks that can be tailored to solve specific or general problems,  learn/discover an optimal distribution network configuration for performing certain types of tasks, or that can contain sufficient redundancy to provide execution within required timelines. Finally, I describe \emph{Diffusive}, a reference implementation in Java, that meets the diffusive programming principles.
\end{abstract}

%
% introduction
%
\section{Introduction}
Task-oriented distributed computing allows independent computational tasks to be distributed to multiple computing nodes, presumably to execute in parallel. This approach can reduce the overall compute time of a set of independent tasks. For example, in quantitative finance, one may desire to calculate risk metrics on a large portfolio of financial instruments. Suppose that the calculation of these risk metrics for each instrument is independent of other instruments. In this case, we can speed up the calculation of the risk metrics for the entire portfolio by spreading these independent tasks over many compute nodes at once, and then collecting the results.

Typically this type of distribution is performed by a distribution middleware. In many cases, the application makes calls to the middleware's application programming interface (API) and implements an interface that represents a compute tasks. Including middleware API calls in the application code couples the application to the middleware. And more concerning, the application is now polluted with code used to distribute execution to remote compute nodes. Clearly, with careful design, much of the distribution logic can be hidden behind wrappers that allow the application logic to interact with an abstract compute engine which could execute locally or remotely. But even that approach doesn't overcome the constraints imposed by requiring that compute tasks implement a specific interface\footnote{An execution method is required. That means that any task must adhere to the signature of the execution method. And conversely, the signature of the execution method must be general enough to support any task.}.

Distribution middleware may also require deployment of certain resources (dynamic libraries, class files, etc) to the compute nodes \emph{before} they can execute remote requests. This requires careful synchronization of versioned resources. And this renders the compute nodes generic, only insofar as the required resources have been deployed to the node.

\emph{Diffusive programming} is an approach for performing task-oriented distributed computing that is based on the following principles intended to overcome challenges described above. In the remainder of this section I describe these principles and define some terms. In the next section, I discuss each principle in more detail. Note that the principles are ordered. The first two principles are related to the action of distributing execution, and the remaining principles are increasingly more focused on the configuration and topology of the compute nodes.

\begin{description}

	% marking, diffusive
	\item[Marking] 
	A method can be marked for remote execution. The act of \emph{marking}, alone, is sufficient and necessary for a method to be executed on a remote location and have the results returned. 
	
	\emph{Definition}: A \textbf{diffusive} method is a method that has been \emph{marked}.

	% location opaquenes
	\item[Location Opaqueness]
	Code calling a \emph{diffusive} method does not, and can not, know on which resource that method was executed. This helps keep code \emph{cohesive} by removing distribution logic from the application. 
	
	\emph{Definition}: A \textbf{diffused} method is a \emph{diffusive} method that was executed. 
	
	\emph{Definition}: A \textbf{diffuser} is what executes \emph{diffusive} method.
	
	% generic computation engine
	\item[Generic Computation]
	Any \emph{diffusive} method can be executed by a \emph{diffuser}. A \emph{diffuser} need not be configured with resources prior to requesting it to execute a method.
	
	% indistinguishablity
	\item[Indistinguishability]
	A \emph{diffuser} is responsible for executing any \emph{diffusive} method, and it is also responsible for \emph{diffusing} methods to other \emph{diffusers}. This implies that there is no distinction between workers and brokers, or clients and servers, in diffusive programming.
	
	% open topology
	\item[Open Topology]
	\emph{Diffusers} can be connected in any topology that can represented as a directed graph. Each node in the directed graph represents a \emph{diffuser}. Each directed edge represents a connection from one \emph{diffuser} to another. The direction of the edge represents the direction of the \emph{diffusion}. And, each \emph{diffuser} may contains connections to a set of other \emph{diffusers}. 
	
	\emph{Definition}: A \emph{diffuser network} is a set of connected \emph{diffusers}.
	
	\emph{Definition}: Suppose we have two \emph{diffusers}, \textbf{A} and \textbf{B}. We say that \textbf{B} is an \emph{end-point} of \textbf{A}, if \textbf{A} \emph{diffuses} methods to \textbf{B}.
	
	This principle allows the construction of networks tailored to solve specific or general problems, networks that can naturally learn/discover an optimal configuration for performing certain types of tasks, or networks that contain sufficient redundancy to provide execution within required timelines.

\end{description}

%
% Principles of diffusive programming
%
\section{Principles of Diffusive Programming}
In this section, I describe the five principles \emph{diffusive programming} in more detail. The principles build on each other, progressing from the action of distributing the code to the deployment and configuration of the \emph{diffusers}. I describe the reason for, and the importance of, each principle. Where applicable, I describe how the principle differs from more typical or traditional methods of distributed computing.

\subsection{Marking}
\emph{Diffusive programming} allows the execution of individual methods to be distributed. \emph{Marking} a method is the act of specifying that a specific method is to be distributed. How a method is \emph{marked} is up to the implementation of this principle. However, this principle does state that the act of \emph{marking} a method is necessary and sufficient for the method to be distributed. This means that \emph{any} method can be marked, and, therefore, executed, regardless of its name, parameters, or return type. This is a departure from many typical task-orient approaches that require the implementation of task interfaces, where the method to be executed has a defined signature (and return type).

The way a method is marked, to become a \emph{diffusive} method, depends on the implementation of this principle. For example, the reference implementation written in Java currently uses annotations to mark methods. However, it could just as easily allow the fully qualified method names to be specified in a configuration file instead.

The act of marking a method decouples the distribution logic from the application logic. And this leads us to the next principle: location opaqueness.

\subsection{Location Opaqueness}
\emph{Marking} a method tells the diffusive framework that that method is to be executed in a distributed manner. But it is the principle of \emph{location opaqueness} that places the requirement that any code calling a diffusive method does not know, or need to know, where that method is executed. Removing the responsibility of knowing or having to deal with the consequences of where the method is executed relieves the calling code of any responsibility regarding distribution. And this allows the application code to remain cohesive. It also means that the same code can be called in a distributed manner, or to run completely locally with any change to the application logic.

In typical distributed systems, the distribution logic must be called directly from the application code. This may occur by calling low level application programming interfaces (API) such as in MPI, or writing task classes that implement interfaces defined by the distribution framework, and then modifying application code to deliver these tasks to the middleware.

Location opaqueness allows code to be endowed with its execution logic, and that execution logic is then automatically mirrored, but in a distributed manner, simply by marking the method(s). When this is coupled with the next principle, \emph{generic computation}, we have a powerful and simple mechanism to distribute computation.

\subsection{Generic Computation}
The principle of \emph{generic computation} provides that any method can be executed on a \emph{diffuser} without out the need to deploy the resource needed to execute that diffused method. Simply put, the shared object libraries or classes don't need to be deployed to the remote server prior to making the request. Each \emph{diffuser} must contain a mechanism for providing resources to remote locations and for loading resources from a remote location.

In typical distributed computing, required resources must be deployed to the remote servers prior to requesting remote execution of a specific task. Diffusive programming removes this restriction by requiring that the mechanism which distributes the method execution also provides a capability to deliver the required resources to execute the method. 

Note that, however, this does not prevent users from deploying resources to a common location from which they can be obtained at run-time. Under certain deployment scenarios, it may be desirable to have such a common location to provide a centralized control over the versions. But even in this case, the resources need only be deployed to the one common area.

\subsection{Indistinguishability}
The principle of \emph{indistinguishability} means that a \emph{diffuser} must be able to receive requests to execute, and at the same time be able to \emph{diffuse} (forward) those requests to another \emph{diffuser}. In other words, there isn't such a thing as a client diffuser and a server diffuser: they are one and the same.

The \emph{generic computation} principle alluded to this principle of \emph{indistinguishability}. The generic computation principle states that a diffuser must be able to load resources from a remote diffusers, \textbf{and} at the same time must be able to provide resources to a remote diffuser.

\subsection{Open Topology}
The \emph{open topology} principle, coupled with the \emph{indistinguishability} principle, requires that it is possible to create networks of \emph{diffusers}, called \emph{diffuser networks}, in any topology that can be represented as a directed graph. Each node in the directed graph represents a \emph{diffuser}, and each (directed) edge connects that \emph{diffuser} to an \emph{end-point}, which is another \emph{diffuser}. Any network that can be represented by a directed graph can be constructed. Section \ref{sec:diffusion_patterns} (\textbf{Diffusion Patterns}) describes a few possible network topologies (patterns) that are designed to solve specific problems.

%
% Diffusion patterns
%
\section{Diffusion Patterns}\label{sec:diffusion_patterns}
\emph{Diffusion patterns} are made possible by the last three diffusive programming principles: generic computing; indistinguishability, and open topology. Together, these three principles make the statement that \emph{diffusers} form building blocks that can be connected as directed graphs. The principle of \emph{open topology} requires that a diffuser be connected to zero or more end-points to which it can diffuse execution of a method. The principle of \emph{indistinguishability} states that each end-point, itself, must be a diffuser. This means that that diffuser itself is connected to zero or more other end-points. And, therefore, it can receive tasks as well as diffuse them. Finally, the principle of \emph{generic computing} requires that a diffuser be able to receive (and send) resources needed to execute a task, allowing tasks to be diffused to other nodes dynamically and executed.

Because any diffuser network that can be represented as a directed graph is possible to construct, there are an infinite number of patters in which these networks can be constructed. And furthermore, their dynamic nature also allows the networks to evolve over time. In the next subsections I describe three illustrative patterns that solve specific problems.

\subsection{Layered}


\subsection{Redundant}

\subsection{Learning}

%
% Reference implementation
%
\section{Diffusive: Reference Implementation}

%
% Future work
%
\section{Future Work}

%Before describing these principles and their implications in more detail, I illustrate how such an approach could behave. First you would create a \emph{diffuser network} by running \emph{diffusers} on a collection of servers. Then you would configure the \emph{diffuser network} by giving each \emph{diffuser} a set of \emph{end-points} to which it can \emph{diffuse} method execution. Then you would specify, through configuration, which methods should be diffused. Then you need to instrument the code so that when the specified methods are called, the method calls are intercepted, and diverted to a \emph{diffuser}, which executes the method and returns the result, or diffuses the method further.

%Although this may seem complicated, a framework constructed from the above principles, would only require that the \emph{diffusive} methods be specified, and the code could be launched through a container that would automatically instrumented the code to divert method calls. In this way, there isn't any need to modify the base code.

%I have developed an open-source reference implementation of a Diffusive framework in Java that meets the above principles. Through the use of Aspect-Oriented programming frameworks, one could also implement such a framework in C/C++.

%Suppose we needed to perform 800 calculations, each calculation tools about 1 minute to execute. Suppose further that we had written code to run on a dual-quad-core server. And, on this hardware it took about 2 hours to execute. Suddenly, we are given access to 100 similar servers and asked to distribute the calculation across these 100 servers to reduce the computation time to a few minutes.

%Task-oriented distributed computing is difficult. Endowing applications with the ability to execute tasks remotely, typically requires calls to an application programming interface (API) that manages the distribution of those tasks to remote servers. Typically, these servers must be configured to run the desired tasks.

%Coding to such an API pollutes the application's \emph{business} logic with its distribution logic. It is true that with care the business and distribution logic can be cleanly separated in most cases. However, when an application needs to provide the ability to execute its tasks both locally and in a distributed mode, applications will require two versions of the execution logic: one for running locally, and one for distributed execution.

%Embedding execution distribution code into the application, in any case, causes additional difficulties for testing the business logic and debugging. 

% reference impl�mentation
%This reference implementation allows developers to simply annotate methods whose execution is to be distributed. Then they launch the application through a \emph{diffusive launcher} that instruments the code at load-time, replacing the annotated method calls with ones that distributes the execution.

%\subsection{}



\end{document}  